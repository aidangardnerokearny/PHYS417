\documentclass{article}
\usepackage{graphicx} % Required for inserting images
\usepackage{amsmath}
% \usepackage{vartau}
\newcommand{\p}[1]{\left(#1\right)}
\newcommand{\braket}[1]{\langle#1\rangle}
\newcommand{\bra}[1]{|#1\rangle}
\newcommand{\ket}[1]{\langle#1|}

\title{PHYS417HW5}
\author{Aidan Gardner-O'Kearny}
\date{May 2025}

\begin{document}

\maketitle

\section*{1)}
\subsection*{2)}

If we're given some vector potential $\vec{A}$ we can show that the resulting magnetic field $\vec{B}$ is gauge invariant; $\vec{B}=\vec{\nabla}\times\vec{A}=\vec{\nabla}\times\vec{A}'$.

We are given the two potentials

\begin{align*}
    \vec{A}&=\frac{B_0}{2}\p{-y\vec{e}_x+x\vec{e}_y}\quad\text{and}\\[1em]
    \vec{A}'&=B_0y\vec{e}_x
\end{align*}

To find $\vec{B}$ we simply take the curl of $\vec{A}$ and $\vec{A}'$.

\begin{align*}
    \vec{\nabla}\times\vec{A}&=\frac{B_0}{2}
    \begin{vmatrix}
        \vec{e}_x &\vec{e}_y &\vec{e}_z \\
        \partial_x &\partial_y &\partial_z \\
        -y &x &0
    \end{vmatrix}\\[1em]
    &=\frac{B_0}{2}\p{\vec{e}_z+\vec{e}_z}\\
    &=B_0\vec{e}_z
\end{align*}

and for $\vec{A}'$

\begin{align*}
    \vec{\nabla}\times\vec{A}&=B_0
    \begin{vmatrix}
        \vec{e}_x &\vec{e}_y &\vec{e}_z \\
        \partial_x &\partial_y &\partial_z \\
        0 &x &0
    \end{vmatrix}\\[1em]
    &=B_0\p{\vec{e}_z}\\
    &=B_0\vec{e}_z
\end{align*}

\subsection*{b)}
More generally, we can say that $\vec{A}'=\vec{A}+\nabla\Lambda$ for some function $\Lambda(\vec{r)}$. This will only be true if $\Lambda(\vec{r})$ is curless, or, in other words, $\vec{\nabla}\times\vec{\nabla}\Lambda=0$.

\begin{align*}
    \vec{\nabla}\times\vec{\nabla}\Lambda&=\begin{vmatrix}
        \vec{e}_x &\vec{e}_y &\vec{e}_z \\
        \partial_x &\partial_y &\partial_z \\
        \partial_x\Lambda & \partial_y\Lambda &\partial_z\Lambda
    \end{vmatrix}\\[1em]
    &=\vec{e}_x\p{\partial_y\partial_z\Lambda-\partial_z\partial_y\Lambda}-\vec{e}_y\p{\partial_x\partial_z\Lambda-\partial_z\partial_x\Lambda}+\vec{e}_z\p{\partial_x\partial_y\Lambda-\partial_y\partial_x\Lambda}\\[1em]
    &=0
\end{align*}
Thus we can pick essentially any $\Lambda(\vec{r})$ we want, so long as it is possible to take the gradient of it. 

\subsection*{c)}
The addition of a $\vec{\nabla}\Lambda(\vec{r})$ term corresponds to the addition of a $e^{i\phi(\vec{r})}$ term to $\bra{\psi}$. I'm going to make a dimensionality based argument for the relation. $\phi(\vec{r})$ must be a unitless quantity. $\vec{A}$ has units of $\frac{\text{momentum}}{\text{charge}}$. Consequently $\Lambda$ has units of $\frac{\text{length}\times\text{momentum}}{\text{charge}}$. $p\times\ell=\frac{E}{s}$, which conveniently has the same units as $\hbar$. To that end the relationship between our choice of gauge and our state's phase is;

\begin{align*}
    \phi=\frac{q}{\hbar}\Lambda
\end{align*}

The book states that 

\begin{align*}
    g(\vec{r})=\frac{q}{\hbar}\int_{\mathcal{O}}^rd\vec{r}'\cdot{}\vec{A}(\vec{r'})
\end{align*}

Which, given that $\vec{A}(\vec{r})=\vec{A}'(\vec{r})+\nabla\Lambda(\vec{r})$, the vector potential term will drop out and we'l only be left with

\begin{align*}
    g(\vec{r})&=\frac{q}{\hbar}\int_{\mathcal{O}}^rd\vec{r}'\cdot{}\nabla\Lambda(\vec{r'})\quad\text{Which by the fundamental theorem for line integrals}\\[1em]
    &=\frac{q}{\hbar}\Lambda(\vec{r})
\end{align*}

\subsection*{d)}
The difference between the cannonical momentum $\braket{\hat{p}}=\braket{ih\hat{\nabla}}$ and the particle momentum $m\frac{d\braket{r}}{dt}$ will be that $\braket{\hat{p}}$ will pull down a derivative of $\Lambda(\vec{r})$ from the exponent, making it so that it \textit{will} depend on the gauge, or more accurately, a derivative of the gauge. $\braket{r}$ on the other hand will not act on the phase term at all, allowing them to cancel out as we would expect $|e^{i\phi}|^2=1$. In this way the classical momentum will be gauge invariant.


\section*{2)}
Preparing a state $\bra{\Psi}=\frac{1}{\sqrt3}[\bra{A}+\bra{B}+\bra{C}]$ to determine which box ($A,B,\text{ or},C$) a particle will end up in after the following measurements:

\begin{align*}
    \hat{A}&=\bra{A}\ket{A}=\bra{B}\ket{B}+\bra{C}\ket{C}\quad\text{or}\\[1em]
    \hat{B}&=\bra{B}\ket{B}=\bra{A}\ket{A}+\bra{C}\ket{C}
\end{align*}

After this I perform someas yet undetermined measurement that projects us into the state

\begin{align*}
    \bra{\Phi}=\frac{1}{\sqrt{3}}[\bra{A}+\bra{B}-\bra{C}]
\end{align*}

The fact that we end up in this state allows us to figure out which measurement you did.


\subsection*{a)}

% If you start by measuring box $A$, there is a $\frac{1}{3}$ chance of the particle being there. We are left with the resulting state $\bra{\Psi}=\frac{1}{\sqrt{2}}[\bra{B}+\bra{C}]$.

There are $4$ potential outcomes from your measurement, if you perform the first measurement, you could either end up in the state $\bra{\psi}=\bra{A}$ or $\bra{\psi}=\frac{1}{\sqrt{2}}[\bra{B}+\bra{C}]$. Similarly, if you performed the second measurement you could end up in the state $\bra{\psi}=\bra{B}$ or $\bra{\psi}=\frac{1}{\sqrt{2}}[\bra{A}+\bra{C}]$.

We then perform another measurement, which projects us into $\bra{\Phi}=\frac{1}{\sqrt{3}}[\bra{A}+\bra{B}-\bra{C}]$. 

The overlap of our new state with the state that resulted from your measurement $\bra{\psi}_{prev}$ will be a coefficient that describes the probability that that specific transition happened.

Starting with the case where you measured box $A$

\begin{align*}
    \braket{\Phi|\psi_{prev}}&=\frac{1}{\sqrt{3}}\p{\braket{A|A}+\braket{B|0}-\braket{C|0}}=\frac{1}{\sqrt{6}}\quad\text{or}\\[1em]
    &=\frac{1}{\sqrt{6}}\p{\braket{A|0}+\braket{B|B}-\braket{C|C}}=0
\end{align*}
! So if you measured $A$ and we were able to get into state $\bra{\Phi}$ then the result of your measurement must have been that the particle was in box $A$.

Similar logic holds if you measure $\bra{B}\ket{B}$.

\begin{align*}
    \braket{\Phi|\psi_{prev}}&=\frac{1}{\sqrt{3}}\p{\braket{A|0}+\braket{B|B}-\braket{C|0}}=\frac{1}{\sqrt{6}}\quad\text{or}\\[1em]
    &=\frac{1}{\sqrt{6}}\p{\braket{A|A}+\braket{B|0}-\braket{C|C}}=0
\end{align*}

So then, we know what outcome was obtained if we know what measurement you did. If you measured $\bra{A}\ket{A}$ you found the particle in that box, and if you measured $\bra{B}\ket{B}$ you found the particle in \textit{that} box.

\subsection*{b)}
This is kind of a weird result. In a classical world the particle would simply \textit{be} in one of the boxes, and we wouldn't be able to do this sort of manipulation. We're doing a measurement that tells us something about the outcome of a previous measurement. The idea that you had to have found the particle in whichever box you looked in has no real classical analogue.

\section*{3)}
Weird stuff with a spin-$\frac{3}{2}$ particle.

\subsection*{a)}
Measuring $S_z^2$ will result in a measurement of the "length" of spin in the $z$ direction. We did something similar to this for the spin-$\frac{1}{2}$ case last term.

A measurement of $S_z$ will return $\pm\frac{3}{2},\pm\frac{1}{2}$ and so a measurement of $S_z^2$ will return $\frac{9}{4},\frac{1}{4}$.

If we measured $S_x^2$ or $S_y^2$ we could return the same values, as those operators will have the same eigenvalues.

\subsection*{b)}

Classically, the sum of all three would be $\frac{27}{4},\frac{19}{4},\frac{11}{4},\text{or }\frac{3}{4}$, assuming that each direction returned one of its two possible eigenvalues.


\subsection*{c)}

But of course, $S^2$ can only return $s(s+1)$, or, in our case $\frac{3}{2}(\frac{3}{2}+1)=\frac{15}{4}$, which is different than any of the values we found in the classical case.

\end{document}
