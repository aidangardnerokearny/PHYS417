\documentclass{article}
\usepackage{graphicx} % Required for inserting images

\title{PHYS 417 HW 2}
\author{Aidan Gardner-O'Kearny}
\date{April 2025}
\usepackage{amsmath}
% \usepackage{vartau}
\newcommand{\p}[1]{\left(#1\right)}
\newcommand{\braket}[1]{\langle#1\rangle}
\newcommand{\bra}[1]{|#1\rangle}
\newcommand{\ket}[1]{\langle#1|}

\begin{document}

\maketitle

\section*{1)}

We have the Hamiltonian

\begin{align*}
    \hat{H}=-\frac{\hbar^2}{2m}\frac{d^2}{dx^2}+\frac{1}{2}k(t)x^2\quad\text{With}\\[1em]
    k(t)=\begin{cases}
        k_0+\epsilon{k_0}\sin(\Omega{t})\quad&{t>0}\\[1em]
        k_0\quad&{t\leq0}
    \end{cases}
\end{align*}

We are given eigenenergies for our unperturbed Hamiltonian of the form;

\begin{align*}
    E_{n}=\hbar\omega\p{n+\frac{1}{2}}
\end{align*}

\subsection*{a)}
At $t=0$ $k(t)=k_0$. At that time it's just a QHO so we can relate $\omega, k_0$ and $m$.

\begin{align*}
    k_0&=m\omega^2\\[1em]
    \omega&=\sqrt{\frac{k_0}{m}}
\end{align*}

% Or, for $t>0$
% \begin{align*}
%     \omega&=\sqrt\frac{k_0\p{1+\epsilon\sin\p{\Omega{t}}}}{m}
% \end{align*}

% Which I'm just now noticing is true \textit{at} $t=0$ as well.

\subsection*{b)}
To rewrite $\hat{H_0}$ in terms of the ladder operators, we just do the following;

\begin{align*}
    \hat{H}_0&=\frac{\hat{p}^2}{2m}+\frac{1}{2}m\omega^2\hat{x}^2\\[1em]
    &=\frac{1}{2m}\p{-\frac{\hbar{m}\omega}{2}\p{\hat{a}_{+}^2+\hat{a}_{-}^2-\hat{a}_{+}\hat{a}_{-}-\hat{a}_{0}\hat{a}_{+}}}+\frac{m\omega^2}{2}\p{\frac{\hbar}{2m\omega}\p{\hat{a}_{+}^2+\hat{a}_{-}^2+\hat{a}_{+}\hat{a}_{-}+\hat{a}_{0}\hat{a}_{+}}}\\[1em]
    &=\frac{\hbar\omega}{4}\p{2\hat{a}_{+}\hat{a}_{-}+2\hat{a}_{-}\hat{a}_{+}}\quad[\hat{a}_{-},\hat{a}_{+}]=1\\[1em]
    &=\hbar\omega\p{\frac{1}{2}\p{\hat{a}_{+}\hat{a}_{-}+\p{\hat{a}_{+}\hat{a}_{-}+1}}}\\[1em]
    &=\hbar\omega\p{\hat{a}_{+}\hat{a}_{-}+\frac{1}{2}}\\[1em]
    % &=\hbar\sqrt{\frac{k_0}{m}}\p{\hat{a}_{+}\hat{a}_{-}+\frac{1}{2}}
\end{align*}

So when it comes time to write $\hat{H}'(t)$ in terms of the ladder operators;

\begin{align*}
    \hat{H}'(t)&=\frac{1}{2}\epsilon{k_0}\sin\p{{\Omega{t}}}\hat{x}^2\\[1em]
    &=\frac{\epsilon\hbar{k}_0}{4mw}\sin(\omega{t})(\hat{a}_{+}^2+\hat{a}_{-}^2+\hat{a}_{+}\hat{a}_{-}+\hat{a}_{-}\hat{a}_{+})\\[1em]
    &=\frac{\epsilon\hbar\omega}{4}\sin(\omega{t})(\hat{a}_{+}^2+\hat{a}_{-}^2+\hat{a}_{+}\hat{a}_{-}+\hat{a}_{-}\hat{a}_{+})
\end{align*}


\subsection*{c)}

Starting in the state $|0\rangle$ and then evolving to some time $t_0$, we are now allowed to exist in other states, with matrix elements; 

% \begin{align*}
%     c_n'(t)&=\frac{i}{\hbar}\int_{0}^{t}dt'\:e^{i\omega{t'}}\braket{n|H'(t')|m}\\[1em]
%     &=\int_{0}^{t}dt'\:i\omega{e}^{i\omega{t'}}\braket{n|\p{\p{\hat{a}_+\hat{a}_-+\frac{1}{2}}+\p{\hat{a}_{+}^2+\hat{a}_{-}^2+\hat{a}_{+}\hat{a}_{-}+\hat{a}_{-}\hat{a}_{+}}}|m}\\[1em]
%     % &=\int_{0}^{t}dt'\:i\omega{e}^{i\omega{t}'}\p{\frac{1}{2}+\sqrt{\p{m^2+m}}}{\braket{n|m}}
% \end{align*}

\begin{align*}
    V_{nm}&={\hbar\omega}\braket{n|\p{\p{\hat{a}_+\hat{a}_-+\frac{1}{2}}+\frac{\epsilon}{4}\p{\hat{a}_{+}^2+\hat{a}_{-}^2+\hat{a}_{+}\hat{a}_{-}+\hat{a}_{-}\hat{a}_{+}}}|m}
\end{align*}

We can say that the $\hat{a}_{+}\hat{a}_-$ will only contribute nonzero terms when $m=n$. We can ignore those as we are only interested in transitions. The only two other terms are an $\hat{a}_+^2$ term and a $\hat{a}_-^2$ term. This implies you can go up and down two states at a time.

To figure out the probabilities of jumping to these states, we can just use Griffiths eq 11.35

\begin{align*}
    |c_n(t)|^2\approx\frac{|V_{nm}|^2}{\hbar^2}\frac{\sin^2\p{[\omega_0-\Omega]\frac{t}{2}}}{\p{\omega_0-\Omega}^2}
\end{align*}

Where $\omega_0$ is found from the energy difference of the two states. To go from $\bra{0}\rightarrow\bra{2}$ (and indeed for any transition in this system), the energy gap is $2\hbar\omega$, implying $\omega_0=2\sqrt{\frac{k_0}{m}}$

Our matrix element $V_{20}$ (the only transition allowed starting from the ground state), will be

\begin{align*}
    V_{20}&=\frac{\epsilon\hbar\omega}{4}\braket{2|\hat{a_+^2}|0}\\[1em]
    &=\frac{\epsilon\hbar\omega}{4}\p{\sqrt{1}\sqrt{2}}\\[1em]
    &=\frac{\sqrt2\epsilon\hbar\omega}{4}
\end{align*}

Which means our probability is;

\begin{align*}
    |c_2(t)|^2\approx\frac{\epsilon^2k_0}{8m}\frac{\sin^2\p{[2\sqrt{\frac{k_0}{m}}-\Omega]\frac{t}{2}}}{\p{2\sqrt{\frac{k_0}{m}}-\Omega}^2}
\end{align*}

\subsection*{d)}

When we start in state $\bra{3}$ we can go up to $\bra{5}$ or down to $\bra{1}$. The respective matrix elements are;

\begin{align*}
    V_{53}&=\frac{\epsilon\hbar\omega}{4}\braket{5|\hat{a}_{+}^2|3}\\[1em]
    &=\frac{\epsilon\hbar\omega}{4}\p{\sqrt{4}\sqrt{5}}\\[1em]
    &=\frac{\sqrt{5}\epsilon\hbar\omega}{\sqrt4}
\end{align*}

and 

\begin{align*}
    V_{13}&=\frac{\epsilon\hbar\omega}{4}\braket{1|\hat{a}_{-}^{2}|3}\\[1em]
    &=\frac{\epsilon\hbar\omega}{4}\p{\sqrt{2}\sqrt{3}}\\[1em]
    &=\frac{\sqrt{6}\epsilon\hbar\omega}{4}
\end{align*}

So thus the respective probabilities are going to be;

\begin{align*}
    |c_5(t)|^2\approx\frac{5\epsilon^{2}{k_0}}{4m}\frac{\sin^2\p{[2\sqrt{\frac{k_0}{m}}-\Omega]\frac{t}{2}}}{\p{2\sqrt{\frac{k_0}{m}}-\Omega}^2}
\end{align*}

and

\begin{align*}
    |c_{1}(t)|^2\approx\frac{3\epsilon^2k_0}{8m}\frac{\sin^2\p{[2\sqrt{\frac{k_0}{m}}-\Omega]\frac{t}{2}}}{\p{2\sqrt{\frac{k_0}{m}}-\Omega}^2}
\end{align*}

\subsection*{e)}

To make this transition more efficient, our probabilities blow up when the denominator is very small. Using $\Omega$ as a lever, we can see that the denominator gets very small when $\Omega\approx2\sqrt{\frac{k_0}{m}}$.

\section*{2)}

Our perturbation is of the form

\begin{align*}
    \hat{H}'(t)=C\delta(x_1-x_2)e^{-t/\tau}
\end{align*}

\subsection*{a)}

The $\delta$-function spits out something with units of $\frac{1}{\text{Length}}$ (discussed this on the last homework), so C must have units of $\text{Energy}\times\text{Length}$.

\subsection*{b)}

To calculate a probability, we say that, if we start in $\bra{1,1}$ (our ground state), and want to see if we evolve into $\bra{2, 2}$.

We are going from state

\begin{align*}
    \psi&=\psi_1\psi_1=A\sin^2\p{\frac{\pi}{a}x}
\end{align*}

to state

\begin{align*}
    \psi&=\psi_2\psi_2=A\sin^2\p{\frac{2\pi}{a}x}
\end{align*}

With the total Hamiltonian

\begin{align*}
    \hat{H}(t)=\frac{\hat{p}^2}{2m}+C\delta(x_1-x_2)e^{-t/\tau}
\end{align*}

The energy of the two states are $E_{11}=\frac{\pi^2\hbar^2}{ma^2}$ and $E_{22}=\frac{4\pi^2\hbar^2}{ma^2}$ Which means $\hbar\omega_0=\frac{3\pi\hbar^2}{ma^2}$ and thus $\omega_0=\frac{3\pi\hbar}{ma^2}$

This means we can calculate our coefficient $c_{22}'(t)$

\begin{align*}
    c_{22}'(t)&=\frac{-i}{\hbar}\int_{0}^{t}dt'\:e^{i\omega{t'}}\braket{\psi_{22}|H'(t)|\psi_{11}}\\[1em]
    &=\frac{-i}{\hbar}\int_{0}^{t}dt'\:e^{i\omega{t'}}\int_{-a}^{a}dx\:\psi_{22}^{\star}H'(t)\psi_{11}
\end{align*}

That position integral is going to get its own section, here;


\begin{align*}
    &\int{dx}\:\psi_{22}^{\star}H(t)\psi_{11}\\[1em]
    &\int_{-a}^{a}dx_{2}\:\int_{-a}^{a}dx_{1}\:\sin(\frac{2\pi}{a}x_1)\sin(\frac{2\pi}{a}x_2)\p{C\delta(x_1-x_2e)e^{-t/\tau}}\sin(\frac{\pi}{a}x_1)\sin(\frac{\pi}{a}x_2)\\[1em]
    &C\int_{-a}^{a}dx_{2}\:\sin^2(\frac{2\pi}{a}x_2)\sin^2(\frac{\pi}{a}x_2)e^{-t/\tau}\\[1em]
    &\frac{Ca}{2}e^{-t/\tau}\quad\text{Thank you interal-calculator.com}
\end{align*}

Now back to the main intergral.

\begin{align*}
    &=-\frac{iCa}{2\hbar}\int_{0}^{t}dt'e^{i\omega_0t'}e^{-t'/\tau}\\[1em]
    &=-\frac{iCa}{2\hbar}\int_{0}^{t}dt'e^{-\frac{i\omega_0\tau-1}{\tau}t'}\\[1em]
    &=\frac{iCa}{2\hbar}\frac{1-i\omega_0\tau}{\tau}\p{e^{-\frac{i\omega_0\tau-1}{\tau}t}-1}
\end{align*}

So $|c_{22}(t)|^2$ is;

\begin{align*}
    &\frac{C^2a^2}{4\hbar^2}\frac{1+\omega_0^2\tau^2}{\tau^2}\p{2-e^{-\frac{i\omega_0\tau-1}{\tau}t}-e^{\frac{i\omega_0\tau-1}{\tau}t}}\\[1em]
    &\frac{C^2a^2}{2\hbar^2}\frac{1+\omega_0^2\tau^2}{\tau^2}\p{1-\cos(\frac{\omega_0\tau-1}{\tau}t)}
\end{align*}

This is awful to look at, but okay.

\section*{3)}
\subsection*{a)}
The ratio of the two decay rates $\Gamma$ as a function of $\omega$ is the ratio of the coefficients $A$ and $R_{b\rightarrow{a}}$.

\begin{align*}
    A&=\frac{\omega_0^3}{3\pi\epsilon_0\hbar{c}^3}|q\braket{\psi_b|\boldsymbol{r}|\psi_a}|^2\\[1em]
    R_{b\rightarrow{a}}&=\frac{\pi}{3\epsilon_0\hbar^2}|q\braket{\psi_b|\boldsymbol{r}|\psi_a}|^2\rho(\omega_0)\\[1em]
    % \Gamma&=\frac{\omega_0^3\hbar}{\pi^2c^3\rho(\omega_0)}
    \Gamma&=\frac{A}{B\rho(\omega_0)}
\end{align*}

This relationship is *cubic* in $\omega_0$, but also proportional to $1/c^3$. The temperature dependence will come from the term $\rho(\omega_0)$.

\begin{align*}
    \rho(\omega_0)&=\frac{A}{B}\frac{1}{e^{\hbar\omega_0/k_BT}-1}
\end{align*}

Which means that

\begin{align*}
    \Gamma&=\frac{1}{e^{\hbar\omega_0/k_BT}-1}
\end{align*}

\subsection*{b)}
For this ratio to be 1

\begin{align*}
    1&=\frac{1}{e^{\hbar\omega_0/k_BT}-1}\\[1em]
    {e^{\hbar\omega_0/k_BT}-1}&=1\\[1em]
    {e^{\hbar\omega_0/k_BT}}&=2\\[1em]
    \frac{\hbar\omega_0}{k_BT}&=\ln(2)\\[1em]
    \omega_0&=\ln(2)\frac{k_BT}{\hbar}  
\end{align*}

Room temperature is $\approx300K$, which puts $\omega_0\approx10^{12}\text{ Hz}$, which puts us well into the infrared part of the spectrum.

At visible light ranges $\omega_0\approx10^{15}\text{ Hz}$. The ratio of $A/B$ will be very low, as the exponent's argument $\omega_0\hbar/k_BT$ will be very large ($\approx160$). This indicates that the rate of stimulated emission will dominate over spontaneous emission.

\section*{4)}
Let's say we start an atom in the state $|n\ell{m}\rangle=|411\rangle$.

\subsection*{a)}
If we consider spontaneous emission of dipole photons, we can have the following decay paths. Based on the rules

\begin{align*}
    \Delta\ell_{}=\ell'-\ell=\pm1\\[1em]
    \Delta{m}=m'-m=0\text{ or }\pm1
\end{align*}

We do have to also consider the old rules we have for orbital (Namely $\ell\geq{|m|}$ and of course $n>\ell$)

The results of these rules can be seen from Griffiths Figure 11.9. The only way to go from $\bra{411}$ to $\bra{100}$ is via a direct jump or via a 3 step process $n=4\rightarrow3\rightarrow2\rightarrow1$

Any even number of jumps (and thus photon emissions), is impossible. That would require a jump from either $\bra{31x}$ or $\bra{21x}$, both of which are impossible to get to in one step from $\bra{411}$.

\subsection*{b)}

If we emit 1 photon (jumping immediately from $\bra{411}$ to $\bra{100}$), $\omega_0=-13.6\text{ eV}\p{\frac{1}{1^2}-\frac{1}{4^2}}=\frac{-13.6\times15}{16\hbar}$. We already know our matrix element so we can siply do the integral;

(Actually a short side note, I decided after doing the integral and looking at the final answer that I will approximate all energies as being $\omega_0=\frac{-13.6}{\hbar}$, the $ea_0$ term is going to so heavily dominate that it shouldn't matter.

\begin{align*}
    c_1(t)&=-\frac{iea_0}{\hbar}\int_{0}^{t}dt'\:e^{i\omega_0{t}'}\\[1em]
    &=-\frac{ea_0}{\hbar\omega_0}\p{e^{i\omega_0t'}|_0^{t}}\\[1em]
    &=-\frac{ea_0}{\hbar\omega_0}\p{e^{i\omega_0t}-1}
\end{align*}

So $|c_n(t)|^2$ is going to be porportional to;
    
\begin{align*}
    \frac{e^2a_0^2}{\hbar^2\omega_0^2}\approx\frac{{a_0^2}}{13.6^2}\approx1.47\times10^{-23}
\end{align*}

Everything cancled out relaly nicely.

% \begin{align*}
%     |e^{-i\omega_0t}-1|^2&=2-e^{-i\omega_0t}-e^{-i\omega_0t}\\[1em]
%     &=2-2\cos\p{\omega_0t}\\[1em]
% \end{align*}

But the other way we can do this is a three step process. That means that our total probability amplitude for that is $|c_3|^2|c_2|^2|c_1|^2$, which will result in a coefficient that is

\begin{align*}
    \frac{e^6a_0^6}{\hbar^6\omega_{43}^2\omega_{32}^2\omega_{21}^2}
\end{align*}

The factors of $\hbar$ will cancel out, but the factors of $e\alpha_0$ will not. Those are both very small numbers, and cubing them will seriously surpress that path from happening. For those reasons, the direct path is more likely.

The other way we can do it (three steps), will be what we found, but cubed. $\approx3.21\times10^{-69}$

\end{document}
